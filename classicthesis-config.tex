%%%%%%%%%%%%%%%%%%%%%%%%%%%%%%%%%%%%%%%%%
% Classicthesis Typographic Thesis
% Configuration File
%
% This file has been downloaded from:
% http://www.LaTeXTemplates.com
%
% Original author:
% André Miede (http://www.miede.de) with extensive commenting changes by:
% Vel (vel@LaTeXTemplates.com)
%
% License:
% GNU General Public License (v2)
%
% Important note:
% The main lines to change in this file are in the DOCUMENT VARIABLES
% section, the rest of the file is for advanced configuration.
%
%%%%%%%%%%%%%%%%%%%%%%%%%%%%%%%%%%%%%%%%%

%----------------------------------------------------------------------------------------
%	CHARACTER ENCODING
%----------------------------------------------------------------------------------------

\PassOptionsToPackage{utf8}{inputenc} % Set the encoding of your files. UTF-8 is the only sensible encoding nowadays. If you can't read äöüßáéçèê∂åëæƒÏ€ then change the encoding setting in your editor, not the line below. If your editor does not support utf8 use another editor!
\usepackage{inputenc}

%----------------------------------------------------------------------------------------
%	DOCUMENT VARIABLES
%	Fill in the lines below to enter your information into the thesis template
%	Each of the commands can be cited anywhere in the thesis
%----------------------------------------------------------------------------------------

% Remove drafting to get rid of the '[ Date - classicthesis version 4.0 ]' text at the bottom of every page
\PassOptionsToPackage{eulerchapternumbers,listings,drafting, pdfspacing, subfig,beramono,eulermath,parts}{classicthesis}
% Available options: drafting parts nochapters linedheaders eulerchapternumbers beramono eulermath pdfspacing minionprospacing tocaligned dottedtoc manychapters listings floatperchapter subfig

% \newcommand{\myTitle}{Investigating electron identification and energy regression
%                       in the forward region in ATLAS\xspace}
\newcommand{\myTitle}{\fontfamily{lmr}\selectfont 
                      TITLE \xspace %% Adjust vspace according to the title length.
                     }
\newcommand{\mySubtitle}{\xspace}
\newcommand{\myDegree}{Master of Science (Cand.scient.)\xspace}
\newcommand{\myName}{Your name\xspace}
\newcommand{\myMail}{Your mail\xspace}
\newcommand{\myProf}{Put name here\xspace} % I didnt use this
\newcommand{\myOtherProf}{Put name here\xspace} % I didnt use this
\newcommand{\mySupervisor}{Supervisor name\xspace}
\newcommand{\mySupervisorMail}{Supervisor mail\xspace}
\newcommand{\myFaculty}{Faculty of Science\xspace}
\newcommand{\myDepartment}{Niels Bohr Institute\xspace}
\newcommand{\myUni}{University of Copenhagen\xspace}
\newcommand{\myLocation}{Copenhagen\xspace}
\newcommand{\myTime}{7\ts{th} of Spooktober, 2020\xspace} %September 2015
\newcommand{\myVersion}{version 1.0\xspace}

%----------------------------------------------------------------------------------------
%	USEFUL COMMANDS
%----------------------------------------------------------------------------------------

\newcommand{\ie}{i.\,e.}
\newcommand{\Ie}{I.\,e.}
\newcommand{\eg}{e.\,g.}
\newcommand{\Eg}{E.\,g.}

\newcommand{\ts}{\textsuperscript}

\newcounter{dummy} % Necessary for correct hyperlinks (to index, bib, etc.)
\providecommand{\mLyX}{L\kern-.1667em\lower.25em\hbox{Y}\kern-.125emX\@}
\newlength{\abcd} % for ab..z string length calculation

%----------------------------------------------------------------------------------------
%	PACKAGES
%----------------------------------------------------------------------------------------

\usepackage{lipsum} % Used for inserting dummy 'Lorem ipsum' text into the template

%------------------------------------------------

%\PassOptionsToPackage{ngerman,american}{babel}  % Change this to your language(s)
% Spanish languages need extra options in order to work with this template
%\PassOptionsToPackage{spanish,es-lcroman}{babel}
\usepackage{babel}

%------------------------------------------------			

\usepackage{csquotes}
\PassOptionsToPackage{%
%backend=biber, % Instead of bibtex
backend=bibtex8,bibencoding=ascii,%
language=auto,%
style=numeric-comp,%
%style=authoryear-comp, % Author 1999, 2010
%bibstyle=authoryear,dashed=false, % dashed: substitute rep. author with ---
sorting=nyt, % name, year, title
maxbibnames=10, % default: 3, et al.
%backref=true,%
natbib=true % natbib compatibility mode (\citep and \citet still work)
}{biblatex}
\usepackage{biblatex}
 
 %------------------------------------------------

\PassOptionsToPackage{fleqn}{amsmath} % Math environments and more by the AMS 
 \usepackage{amsmath}
 
 %------------------------------------------------

\PassOptionsToPackage{T1}{fontenc} % T2A for cyrillics
\usepackage{fontenc}

%------------------------------------------------

\usepackage{textcomp} % Fix warning with missing font shapes

%------------------------------------------------

\usepackage{scrhack} % Fix warnings when using KOMA with listings package  

%------------------------------------------------

\usepackage{xspace} % To get the spacing after macros right

%------------------------------------------------

\usepackage{mparhack} % To get marginpar right

%------------------------------------------------

\usepackage{fixltx2e} % Fixes some LaTeX stuff 

%------------------------------------------------

% \PassOptionsToPackage{smaller,withpage}{acronym} % Include printonlyused in the first bracket to only show acronyms used in the text
\usepackage[smaller,withpage]{acronym} % Nice macros for handling all acronyms in the thesis

%\renewcommand*{\acsfont}[1]{\textssc{#1}} % For MinionPro
\renewcommand*{\aclabelfont}[1]{\acsfont{#1}}

%------------------------------------------------

\PassOptionsToPackage{pdftex}{graphicx}
\usepackage{xcolor} % \color{red} to color text
\usepackage{graphicx} % \includegraphics
\usepackage{eso-pic} % \AddToShipoutPicture
\usepackage{hyperref} % Link refferences \href
\usepackage{tcolorbox} % \begin{tcolorbox}  for textboxes
% \usepackage{tufte-sidenotes} % \sidenote{}
\usepackage{sidenotes}
\usepackage{hyperref}
\usepackage{gensymb}
\usepackage{varwidth}
\usepackage{marginnote}
\usepackage{nccmath}
\usepackage{makecell}
\usepackage{braket}
\usepackage{subfig}
\usepackage{slashed}
\usepackage{multirow}
\usepackage{array}
\usepackage{float}
% \usepackage[top=3in]{geometry}

% \usepackage{perpage} %the perpage package
% \MakePerPage{footnote} %the perpage package command
\usepackage[stable, perpage]{footmisc} %,side

\newcommand{\marnote}[1] { \marginnote{\begin{footnotesize} #1 \end{footnotesize}} }
%----------------------------------------------------------------------------------------
%	FLOATS: TABLES, FIGURES AND CAPTIONS SETUP
%----------------------------------------------------------------------------------------

\usepackage{tabularx} % Better tables
\setlength{\extrarowheight}{3pt} % Increase table row height
\newcommand{\tableheadline}[1]{\multicolumn{1}{c}{\spacedlowsmallcaps{#1}}}
\newcommand{\myfloatalign}{\centering} % To be used with each float for alignment
\usepackage{caption}
\captionsetup{font=small}
\usepackage{subfig}  

%----------------------------------------------------------------------------------------
%	CODE LISTINGS SETUP
%----------------------------------------------------------------------------------------

\usepackage{listings} 
%\lstset{emph={trueIndex,root},emphstyle=\color{BlueViolet}}%\underbar} % For special keywords
\lstset{language=[LaTeX]Tex,%C++ % Specify the language(s) for listings here
morekeywords={PassOptionsToPackage,selectlanguage},
keywordstyle=\color{RoyalBlue}, % Add \bfseries for bold  %%Was RoyalBlue
basicstyle=\small\ttfamily, % Makes listings a smaller font size and a different font
%identifierstyle=\color{NavyBlue}, % Color of text inside brackets
commentstyle=\color{Green}\ttfamily, % Color of comments
stringstyle=\rmfamily, % Font type to use for strings
numbers=left, % Change left to none to remove line numbers
numberstyle=\scriptsize, % Font size of the line numbers
stepnumber=5, % Increment of line numbers
numbersep=8pt, % Distance of line numbers from code listing
showstringspaces=false, % Sets whether spaces in strings should appear underlined
breaklines=true, % Force the code to stay in the confines of the listing box
%frameround=ftff, % Uncomment for rounded frame
%frame=single, % Frame border - none/leftline/topline/bottomline/lines/single/shadowbox/L
belowcaptionskip=.75\baselineskip % Space after the "Listing #: Desciption" text and the listing box
}

%----------------------------------------------------------------------------------------
%	HYPERREFERENCES
%----------------------------------------------------------------------------------------

\PassOptionsToPackage{pdftex,pdfpagelabels}{hyperref} %hyperfootnotes=false,
\usepackage{hyperref}  % backref linktocpage pagebackref
\pdfcompresslevel=9
\pdfadjustspacing=1

\hypersetup{
% Uncomment the line below to remove all links (to references, figures, tables, etc), useful for b/w printouts
%draft, 
colorlinks=true, linktocpage=true, pdfstartpage=3, pdfstartview=FitV,
% Uncomment the line below if you want to have black links (e.g. for printing black and white)
%colorlinks=false, linktocpage=false, pdfborder={0 0 0}, pdfstartpage=3, pdfstartview=FitV, 
breaklinks=true, pdfpagemode=UseNone, pageanchor=true, pdfpagemode=UseOutlines,%
plainpages=false, bookmarksnumbered, bookmarksopen=true, bookmarksopenlevel=1,%
hypertexnames=true, pdfhighlight=/O,%nesting=true,%frenchlinks,%
urlcolor=webbrown, linkcolor=RoyalBlue, citecolor=webgreen, %pagecolor=RoyalBlue,%
    %urlcolor=Black, linkcolor=Black, citecolor=Black, %pagecolor=Black,%
%------------------------------------------------
% PDF file meta-information
pdftitle={\myTitle},
pdfauthor={\textcopyright\ \myName, \myUni, \myFaculty},
pdfsubject={},
pdfkeywords={},
pdfcreator={pdfLaTeX},
pdfproducer={LaTeX with hyperref and classicthesis}
%------------------------------------------------
}

%----------------------------------------------------------------------------------------
%	AUTOREFERENCES SETUP
%	Redefines how references in text are prefaced for different 
%	languages (e.g. "Section 1.2" or "section 1.2")
%----------------------------------------------------------------------------------------

\makeatletter
\@ifpackageloaded{babel}
{
\addto\extrasamerican{
\renewcommand*{\figureautorefname}{Figure}
\renewcommand*{\tableautorefname}{Table}
\renewcommand*{\partautorefname}{Part}
\renewcommand*{\chapterautorefname}{Chapter}
\renewcommand*{\sectionautorefname}{Section}
\renewcommand*{\subsectionautorefname}{Section}
\renewcommand*{\subsubsectionautorefname}{Section}
}
\addto\extrasngerman{
\renewcommand*{\paragraphautorefname}{Absatz}
\renewcommand*{\subparagraphautorefname}{Unterabsatz}
\renewcommand*{\footnoteautorefname}{Fu\"snote}
\renewcommand*{\FancyVerbLineautorefname}{Zeile}
\renewcommand*{\theoremautorefname}{Theorem}
\renewcommand*{\appendixautorefname}{Anhang}
\renewcommand*{\equationautorefname}{Gleichung}
\renewcommand*{\itemautorefname}{Punkt}
}
\providecommand{\subfigureautorefname}{\figureautorefname} % Fix to getting autorefs for subfigures right
}{\relax}
\makeatother

%----------------------------------------------------------------------------------------

\usepackage{classicthesis} 

%----------------------------------------------------------------------------------------
%	CHANGING TEXT AREA 
%----------------------------------------------------------------------------------------

%\linespread{1.05} % a bit more for Palatino
%\areaset[current]{312pt}{761pt} % 686 (factor 2.2) + 33 head + 42 head \the\footskip
%\setlength{\marginparwidth}{7em}%
%\setlength{\marginparsep}{2em}%
\addtolength{\topmargin}{10pt} %std: 10pt -> 12pt
\addtolength{\headsep}{23pt} %std: 25pt

%----------------------------------------------------------------------------------------
%	USING DIFFERENT FONTS
%----------------------------------------------------------------------------------------

%\usepackage[oldstylenums]{kpfonts} % oldstyle notextcomp
%\usepackage[osf]{libertine}
%\usepackage[light,condensed,math]{iwona}
%\renewcommand{\sfdefault}{iwona}
%\usepackage{lmodern} % <-- no osf support :-(
%\usepackage{cfr-lm} % 
%\usepackage[urw-garamond]{mathdesign} <-- no osf support :-(
%\usepackage[default,osfigures]{opensans} % scale=0.95 
%\usepackage[sfdefault]{FiraSans}

%----------------------------------------------------------------------------------------
%	SET UP CAPTION SETTINGS
%----------------------------------------------------------------------------------------

\definecolor{nice_red_mahdude}{RGB}{173,49,54}

\captionsetup{
justification=raggedright,
labelfont={color=nice_red_mahdude,bf}, font=footnotesize,
%font=small,textfont=it,
indention=0pt,format=plain}

%----------------------------------------------------------------------------------------
%	SET COLORS
%----------------------------------------------------------------------------------------

